%!TEX program = xelatex

\documentclass[11pt,titlepage]{report}
%!TEX root = main.tex

\usepackage[T1]{fontenc}
\usepackage{lmodern}
\usepackage[svgnames]{xcolor}
\usepackage{fontspec} % XeLaTeX required!
\usepackage{graphicx}
\usepackage{tikz}
\usepackage{pifont}
\usepackage[some]{background}
\usepackage{xltxtra} 
\usepackage{setspace}
\usepackage[absolute]{textpos}
\usepackage[latin1]{inputenc}
\usepackage[english]{babel}
\usepackage{graphicx}
\usepackage{wrapfig}
\usepackage{fullpage}
\usepackage[margin=1in]{geometry}
\usepackage{float}
\usepackage{url}
\usepackage{multicol}
\usepackage{hyperref}
\usepackage{titlepic}
\usepackage{standalone}
\usepackage{siunitx}
\usepackage{booktabs}
\usepackage{amsmath}
\usepackage{unicode-math}
\usepackage{verbatim}
\usepackage{enumitem}
\usepackage{listings}
\usepackage{multirow}
\usepackage{pgfplots}
\pgfplotsset{compat=1.8}
\usepackage{caption} 
\usepackage[parfill]{parskip}
\usepackage{import}
\usepackage[backend=bibtexu,texencoding=utf8,bibencoding=utf8,style=ieee,sortlocale=en_GB,language=auto]{biblatex}
\usepackage[strict,autostyle]{csquotes}
\usepackage{pdfpages}
%\usepackage{enumerate}
%\captionsetup[table]{skip=10pt}

\input{../../library/style}
\addbibresource{../../library/bibliography.bib}

\begin{document}

\chapter{Assignment 2}
\section{Controlling the output}
Consider the state-space model

\begin{align}
	\dot{\vec{x}} &= \mat{A}\vec{x} + \mat{B}\vec{u}, \\
	\vec{y} &= \mat{C} \vec{x}. \label{eq:ass-2-model-output}
\end{align}

Let

\begin{equation}
	\vec{u} = -\mat{K}\vec{x} + \vec{r}
\end{equation}

be a feedback law which renders the considered state-space system asymptotically stable. Substituting $\vec{u}$ yields

\begin{equation} \label{eq:ass-2-derivative}
	\dot{\vec{x}} = (\mat{A} - \mat{B} \mat{K}) \vec{x} + \mat{B} \vec{r}.
\end{equation}

Using the fact that our system is asymptotically stable, we can argue that $\dot{\vec{x}} \to 0$ when $t \to \infty$. Combining Equation~\ref{eq:ass-2-model-output} and \ref{eq:ass-2-derivative} yields

\begin{equation}
	\vec{y} \to \mat{C} (\mat{B} \mat{K} - \mat{A})^{-1} \mat{B} \vec{r} \text{ when } t \to \infty.
\end{equation}

Therefore, the output $\vec{y}$ converges to the scaled applied input $\vec{r}$. We can utilize this fact to control the output of the model.


\end{document}