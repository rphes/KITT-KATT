%!TEX program = xelatex

\documentclass[11pt,titlepage]{report}
%!TEX root = main.tex

\usepackage[T1]{fontenc}
\usepackage{lmodern}
\usepackage[svgnames]{xcolor}
\usepackage{fontspec} % XeLaTeX required!
\usepackage{graphicx}
\usepackage{tikz}
\usepackage{pifont}
\usepackage[some]{background}
\usepackage{xltxtra} 
\usepackage{setspace}
\usepackage[absolute]{textpos}
\usepackage[latin1]{inputenc}
\usepackage[english]{babel}
\usepackage{graphicx}
\usepackage{wrapfig}
\usepackage{fullpage}
\usepackage[margin=1in]{geometry}
\usepackage{float}
\usepackage{url}
\usepackage{multicol}
\usepackage{hyperref}
\usepackage{titlepic}
\usepackage{standalone}
\usepackage{siunitx}
\usepackage{booktabs}
\usepackage{amsmath}
\usepackage{unicode-math}
\usepackage{verbatim}
\usepackage{enumitem}
\usepackage{listings}
\usepackage{multirow}
\usepackage{pgfplots}
\pgfplotsset{compat=1.8}
\usepackage{caption} 
\usepackage[parfill]{parskip}
\usepackage{import}
\usepackage[backend=bibtexu,texencoding=utf8,bibencoding=utf8,style=ieee,sortlocale=en_GB,language=auto]{biblatex}
\usepackage[strict,autostyle]{csquotes}
\usepackage{pdfpages}
%\usepackage{enumerate}
%\captionsetup[table]{skip=10pt}

\input{../../library/style}
\addbibresource{../../library/bibliography.bib}

\begin{document}

\chapter{Assignment 2}
\section{Labday 4}

\subsection{Report 1}
\label{subsec:ass2-report-1}
As calculated in Section~\ref{subsec:ass1-rep-12-13} the sampling frequency should be $F_s=\SI{34}{\kilo\hertz}$ in order to obtain a \SI{1}{\centi\meter} spatial resolution. For our measurements we will use a commonly supported sampling rate of \SI{44.1}{\kilo\hertz}. This sampling rate would then allow for a resolution of $34 / 44.1 = \SI{0.78}{\centi\meter}$. The maximum propagation delay corresponding to \SI{5}{\meter} is $t_{\text{5m}}=5/340=\SI{147}{\milli\second}$. This corresponds with \num{649} samples at \SI{44.1}{\kilo\hertz}, thus defining the peak search interval.

\begin{figure}[H]
	\centering
	\includegraphics[width=0.8\textwidth]{../../deliverable-7-resources/figures/ass-2/report-1/ass-2-report-1.pdf}
	\caption{A typical impulse response with the detected peak.}
	\label{fig:ass-2-rep-1-typical}
\end{figure}

A typical channel impulse response is shown in Figure~\ref{fig:ass-2-rep-1-typical}. We can see that a its duration is about \SI{10}{\milli\second}, before it is no longer distinguishable from noise. Therefore, we chose \texttt{timer3} to be \SI{8}{\hertz} so that the repetition period is \SI{0.125}{\milli\second}, which will fit a single impulse response whilst allowing for some margin. Lastly, we chose \texttt{timer0} to be \SI{15}{kHz} and \texttt{timer1} to be \SI{3}{kHz}, since this frequency range (between \num{12} and \SI{18}{kHz}) is less prone to interference (e.g.from human speech) than lower ranges; and higher ranges were likely not to be transmitted correctly by our beacon, or to be received correctly by the microphones.

%TODO Explanation of peak detection algorithm here. <--NOT NECESSARY??

\subsection{Report 2 and 3}
The TDOA algorithm that we implemented has the following basic procedure: 
\begin{enumerate}
\item Send a signal $x[n]$ through a speaker to two microphones
\item The microphones pick up signals $y_1[n]$ and $y_2[n]$
\item Calculate $h_i[n]$ from $y_i[n]=x[n]*h_i[n]$ 
\item Search for peaks in $h_i[n]$ and determine the number of samples between the peaks in $h_1[n]$ and $h_2[n]$.
\item From the number of samples in between peaks the time delay can be computed (the TDOA) and using the speed of sound the distance between the microphones.
\end{enumerate}
Note that for step three multiple methods are suitable. Three are described in the manual: deconvolution using matrix inversion, a matched filter approach and frequency domain deconvolution. For this report we decided to use the matched filter approach (NOT TRUE) for which the source code may be found in the appendix. %%INSERT SOURCE CODE!
\begin{figure}[H]
	\centering
	\includegraphics[width=0.8\textwidth]{../../deliverable-7-resources/figures/ass-2/report-2-3/ass-2-report-2-typical-data-segment.pdf}
	\caption{Typical received data segment.}
	\label{fig:ass-2-rep-2-data-segment}
\end{figure}

An example of a received data segment is shown in figure \ref{fig:ass-2-rep-2-data-segment}. For details on the peak detection algorithm, refer to Section~\ref{subsec:ass2-report-1}. For an example of the location of the peak as detected by our software see section \ref{subsec:ass2-report-1}.

\begin{figure}[H]
	\centering
	\includegraphics[width=0.8\textwidth]{../../deliverable-7-resources/figures/ass-2/report-2-3/ass-2-report-2-impulse-responses-4.pdf}
	\caption{Comparison of impulse responses for different microphone distances.}
	\label{fig:ass-2-rep-2-impulse-1-20}
\end{figure}

A response of one of our measurements is shown in figure \ref{fig:ass-2-rep-2-impulse-1-20}. This plot is a good example of two microphones placed some distance apart because it is clear the amplitudes of the two responses are about a factor two different. 

\begin{figure}[H]
	\centering
	\includegraphics[width=0.8\textwidth]{../../deliverable-7-resources/figures/ass-2/report-2-3/ass-2-report-2-results.pdf}
	\caption{TDOA estimations for different microphone distances.}
	\label{fig:ass-2-rep-2-result}
\end{figure}

Figure \ref{fig:ass-2-rep-2-result} shows an overview of the TDOA distance estimations for microphones placed \num{0}, \num{10}, \num{20}, \num{50} and \SI{100}{\centi\meter}. Shown in green are the TDOA estimations and shown in red the actual distances between microphones. We can see that for larger spacing between the microphones, the estimation becomes worse. This is to be expected; when the microphones are placed further apart the signal at the farthest microphone will be quieter and it will be increasingly difficult to exactly locate the peak. 

\begin{figure}[H]
	\centering
	\includegraphics[width=0.8\textwidth]{../../deliverable-7-resources/figures/ass-2/report-2-3/ass-2-report-3.pdf}
	\caption{Multiple TDOA estimations for the same microphone placement.}
	\label{fig:ass-2-rep-3}
\end{figure}

An overview of multiple TDOA estimations for the same microphone placement (\SI{10}{\centi\meter} between the microphones) is shown in figure \ref{fig:ass-2-rep-3}. We calculated the average estimation to be \SI{9.38}{\centi\meter} with a standard deviation of \SI{0.35}{\centi\meter}. Because the true distance was \SI{10}{\centi\meter} we see that the measurements have a slight negative bias averaging an error of \SI{-0.62}{\centi\meter}. Also, it is interesting to note that the difference between the minimum and maximum values of the measured distance is equal to \SI{0.78}{cm}, which corresponds perfectly to the value calculated in Section~\ref{subsec:ass2-report-1}.

\subsection{Report 4}
\label{subsec:ass2-report-4}
Plot of typical estimated channel impulse response

\subsection{Report 5}
The general idea of the deconvolution approach is that a signal $y[n]=x[n]*h[n]$ is received, where $h[n]$ is the channel response. The deconvolution method using matrix inversion tells us the channel can be approximated by $\vec{\hat{h}}=(\mat{X}^T\mat{X})^{-1}\mat{X}^T\vec{y}$. Because the sent signal $x[n]$ is known, the matrix $\mat{X}^\dagger=\mat{X}^T\mat{X})^{-1}\mat{X}^T$ can be pre-computed, leaving only a matrix multiplication every time $\vec{\hat{h}}$ must be computed. 

In the method outlined above it is immediately clear that for the same (or similar) $\mat{X}^\dagger$ matrices (and thus $x[n]$ codes) the channel estimation will produce incorrect results. While mathematically not correct, it can be intuitively thought of as follows: the algorithm looks for $x_1[n]$ occurring in $y[n]$. Now if we have a second $x$, $x_2[n]$ and this $x_2[n]$ is similar to $x_1[n]$ then it can be expected that the algorithm mistakes an occurrence of $x_1[n]$ for an occurrence of $x_2[n]$. More mathematically, bladiebla

\subsection{5-channel TCOA measurements}
Read page 106 for more info
\end{document}