%!TEX program = xelatex

\documentclass[11pt,titlepage]{report}
%!TEX root = main.tex

\usepackage[T1]{fontenc}
\usepackage{lmodern}
\usepackage[svgnames]{xcolor}
\usepackage{fontspec} % XeLaTeX required!
\usepackage{graphicx}
\usepackage{tikz}
\usepackage{pifont}
\usepackage[some]{background}
\usepackage{xltxtra} 
\usepackage{setspace}
\usepackage[absolute]{textpos}
\usepackage[latin1]{inputenc}
\usepackage[english]{babel}
\usepackage{graphicx}
\usepackage{wrapfig}
\usepackage{fullpage}
\usepackage[margin=1in]{geometry}
\usepackage{float}
\usepackage{url}
\usepackage{multicol}
\usepackage{hyperref}
\usepackage{titlepic}
\usepackage{standalone}
\usepackage{siunitx}
\usepackage{booktabs}
\usepackage{amsmath}
\usepackage{unicode-math}
\usepackage{verbatim}
\usepackage{enumitem}
\usepackage{listings}
\usepackage{multirow}
\usepackage{pgfplots}
\pgfplotsset{compat=1.8}
\usepackage{caption} 
\usepackage[parfill]{parskip}
\usepackage{import}
\usepackage[backend=bibtexu,texencoding=utf8,bibencoding=utf8,style=ieee,sortlocale=en_GB,language=auto]{biblatex}
\usepackage[strict,autostyle]{csquotes}
\usepackage{pdfpages}
%\usepackage{enumerate}
%\captionsetup[table]{skip=10pt}

\input{../../library/style}
\addbibresource{../../library/bibliography.bib}

\begin{document}

\chapter{Assignment 2}
\section{Labday 4}

\subsection{Report 1}
In order to obtain a spatial resolution of \SI{1}{\centi\meter} the frequency of the audio signal should have a wavelength of at most \SI{1}{\centi\meter}. This means that $f=\frac{v_\text{sound,air}}{\lambda_{signal}}\geq\frac{340}{0.01}=\SI{34}{\kilo\hertz}$, assuming the speed of sound in air can be approximated by \SI{340}{\meter\per\second}. To satisfy the Nyquist criterion, the microphones must sample at twice this frequency to avoid aliasing. This seems a bit high and might lead to problems both in storage of samples (5 channels) and processing power. Therefore, sampling at \SI{44.1}{\kilo\hertz} seems more realistic. The highes frequency that can be detected accurately is then \SI{22.05}{\kilo\hertz}, corresponding to \SI{1.54}{\centi\meter}; this seems reasonable. The maximal propagation delay corresponding to \SI{5}{\meter} is $t_{\text{5m}}=\frac{5}{340}=\SI{147}{\milli\second}$. This corresponds to \num{649} samples at \SI{44.1}{\kilo\hertz}.

%%typical_impulse_response.png included HERE.
%%fig:ass2_typical

A typical channel impulse response is shown in figure \ref{fig:ass2_typical} from which we can see that a typical impulse response takes about \SI{10}{\milli\second} before it is no longer distinguishable from noise. Therefore, we chose \texttt{timer3} to be \SI{8}{\hertz} so that the repetition period is \SI{0.125}{\milli\second}.

\subsection{Report 2}
Apply channel estimation and peak detection algorithms

\subsection{Report 3}
Mean and variance of TDOA estimates

\subsection{Report 4}
Plot of typical estimated channel impulse response

\subsection{Report 5}
Interference with neighbours?

\subsection{5-channel TCOA measurements}
Read page 106 for more info
\end{document}