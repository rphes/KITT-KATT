%!TEX program = xelatex

\documentclass[11pt,titlepage]{report}
%!TEX root = main.tex

\usepackage[T1]{fontenc}
\usepackage{lmodern}
\usepackage[svgnames]{xcolor}
\usepackage{fontspec} % XeLaTeX required!
\usepackage{graphicx}
\usepackage{tikz}
\usepackage{pifont}
\usepackage[some]{background}
\usepackage{xltxtra} 
\usepackage{setspace}
\usepackage[absolute]{textpos}
\usepackage[latin1]{inputenc}
\usepackage[english]{babel}
\usepackage{graphicx}
\usepackage{wrapfig}
\usepackage{fullpage}
\usepackage[margin=1in]{geometry}
\usepackage{float}
\usepackage{url}
\usepackage{multicol}
\usepackage{hyperref}
\usepackage{titlepic}
\usepackage{standalone}
\usepackage{siunitx}
\usepackage{booktabs}
\usepackage{amsmath}
\usepackage{unicode-math}
\usepackage{verbatim}
\usepackage{enumitem}
\usepackage{listings}
\usepackage{multirow}
\usepackage{pgfplots}
\pgfplotsset{compat=1.8}
\usepackage{caption} 
\usepackage[parfill]{parskip}
\usepackage{import}
\usepackage[backend=bibtexu,texencoding=utf8,bibencoding=utf8,style=ieee,sortlocale=en_GB,language=auto]{biblatex}
\usepackage[strict,autostyle]{csquotes}
\usepackage{pdfpages}
%\usepackage{enumerate}
%\captionsetup[table]{skip=10pt}

\input{../../library/style}
\addbibresource{../../library/bibliography.bib}

\begin{document}

\chapter{Homework}
Bandwidth is the range of frequencies which is used to describe a signal. For baseband signals this is the signal's highest frequency. The capacity of a channel is the  theoretical maximum reliable rate of information which can be achieved, in contrary to the data rate, which is the actual rate of information. If the average power of the signal is denoted by $S$, and the average power of the noise by $N$, then the bandwidth $B$ and capacity $C$ are related by the Shannon-Hartley theorem

\begin{equation}
	C=B \log_2{\left(1+\frac{S}{N}\right)}.
\end{equation}
%task 2
Table \ref{tab:pros-cons} shows the pros and cons of digital modulation compared to analog signals.

\begin{table}[H]
	\centering
	\caption{Pros and cons of digital modulation compared to analog signals}
	\label{tab:pros-cons}
	\begin{tabular}{c c}
		\hline\hline
		Analog signals & Digital modulation \\
		\hline
		Smaller bandwidth & Larger bandwidth \\
		Sensitive to noise & Less sensitive to noise \\
		Inflexible electronics & Flexible and cheap electronics \\
		No synchronization required & Synchronization required \\
		No error detection or correction possible & Error detection or correction possible \\
		Multiplexing not possible & Multiplexing possible \\
		Hard to encrypt & Easy to encrypt \\
		\hline
	\end{tabular}
\end{table}

A signal can be assigned to a carrier wave by modulation. This allows for efficient use of the spectrum as each signal is assigned its own range of frequencies. The quality of a signal is often described using its signal-to-noise ratio. The bit error rate is the ratio of received faulty bits to the total bits received, which can be a good estimate of the bit error probability. Causes of bit errors include transmission channel noise, distortion, interference and bit synchronization problems. If we assume that for a FSK signal the frequencies $f_{\text{min}}$ and $f_{\text{max}}$ are sufficiently far apart, then this signal is characterized by a sine with time-varying frequency. The relationship between an ASK signal and BPSK signal is given by $s_{\text{ASK}}(t) = \mu s_{\text{BPSK}}(t) + A \sin{(2 \pi f t)}$. The second term represents the carrier wave.

Transmission of a signal is a tedious and complicated process. The main cause of trouble is noise (e.g. quantization noise, thermal noise), which is introduced in almost every stage. This eventually may result in bit errors, for which error detection and correction may be possible. The general solution to this problem is to use techniques which increase the signal-to-noise ratio. Another important aspect is the sampling process. If the sampling frequency is not at least twice as high as the signal's highest frequency, then aliasing occurs, which introduces even more noise.

\paragraph{Error correcting coding techniques}
%sauce:http://cseweb.ucsd.edu/users/varghese/TEACH/cs123/dl2.pdf
The connection between modulation and coding is that they are both methods to increase the accuracy of transmission over a noisy channel. Coding could also refer to line coding, a method of representing information into different line levels.

Errors can be divided into two general classes; intersymbol interference (e.g. single errors) and noise. Important causes of errors are thermal noise and incorrect synchronization at the physical layer, leading to burst errors (e.g. multiple errors after each other).

When a signal with some form of redundancy is received, the receiver can check if the original data matches with the added redundant data. If this is not the case, we \emph{detect} an error in the received signal. Then, if we have added enough redundant data to the original signal, it might be possible to reconstruct the original signal, so that it matches with the redundant data again. We have then \emph{corrected} the error.

There are two main categories of error-correcting codes; block codes and convolutional codes. The central difference between these two is that block codes are processed block-by-block, and convolutional codes bit-by-bit. Famous examples of block codes are Hamming codes and Reed-Solomon codes. 

There are some important considerations when choosing a coding method; spectrum of coding method must match available baseband channel, available bandwidth is sufficient, availability of self-synchronization, required bit error probability.

Interleaving means you send the information in a `shuffled' order over the line. After transmission the signal is deinterleaved, or placed back in order. The main purpose of this method is burst error correction, since errors occuring adjacent to each other in transmission are unlikely to be adjacent after deinterleaving. Interleaving thus exploits the ease of correcting single errors.

\section{Assignment}
Now, let us perform a calculation of the speed of sound in an environment of \SI{20}{\degree C} and a typical relative humidity of \SI{50}{\%}. Our derivation of the expression to perform this calculation with can be found in Appendix~\ref{app:speed-of-sound}. We take $p_{tot}$ to be \SI{100}{kPa}, the standard atmospheric pressure, and it is given that $R_a = \SI{287.05}{J/kgK}$ and $R_w = \SI{461.495}{J/kgK}$ \cite{sengpiel-sound-speed}. Then we need to find a value for the adiabatic index $\gamma$. $\gamma$ is usually taken to be \num{1.67} for mono-atomic molecules, \num{1.4} for di-atomic molecules and \num{1.33} for tri-atomic molecules. Now, since air mostly consists of nitrogen and oxygen, which are di-atomic molecules, we take $\gamma = 1.4$ \cite{eng-tb-air-comp}. Lastly, we need a value for $e^*_w$. Buck \cite{buck-sat-press} presented an expression to approximate this value, but for this report it will suffice to make use of a table which presents the values of $e^*_w$ for different temperatures. In this manner, we find $e^*_w = \SI{2310}{Pa}$ for $T = \SI{20}{\degree C}$ \cite{eng-tb-sat-press}.
Substitution of the given values in \ref{eq:c2} yields:

\begin{equation}
	c = \sqrt{1.4 \frac{10^4 \cdot 100 \times 10^3 \cdot 287.05 \cdot 461.495 \cdot 293}{100 \cdot 461.495(100 \cdot 100 \times 10^3 - 50 \cdot 2310) + 100 \cdot 287.05 \cdot 50 \cdot 2310}} = \SI{343.90}{m/s}.
\end{equation}

\end{document}