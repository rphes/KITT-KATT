%!TEX program = xelatex

\documentclass[11pt,titlepage]{report}
%!TEX root = main.tex

\usepackage[T1]{fontenc}
\usepackage{lmodern}
\usepackage[svgnames]{xcolor}
\usepackage{fontspec} % XeLaTeX required!
\usepackage{graphicx}
\usepackage{tikz}
\usepackage{pifont}
\usepackage[some]{background}
\usepackage{xltxtra} 
\usepackage{setspace}
\usepackage[absolute]{textpos}
\usepackage[latin1]{inputenc}
\usepackage[english]{babel}
\usepackage{graphicx}
\usepackage{wrapfig}
\usepackage{fullpage}
\usepackage[margin=1in]{geometry}
\usepackage{float}
\usepackage{url}
\usepackage{multicol}
\usepackage{hyperref}
\usepackage{titlepic}
\usepackage{standalone}
\usepackage{siunitx}
\usepackage{booktabs}
\usepackage{amsmath}
\usepackage{unicode-math}
\usepackage{verbatim}
\usepackage{enumitem}
\usepackage{listings}
\usepackage{multirow}
\usepackage{pgfplots}
\pgfplotsset{compat=1.8}
\usepackage{caption} 
\usepackage[parfill]{parskip}
\usepackage{import}
\usepackage[backend=bibtexu,texencoding=utf8,bibencoding=utf8,style=ieee,sortlocale=en_GB,language=auto]{biblatex}
\usepackage[strict,autostyle]{csquotes}
\usepackage{pdfpages}
%\usepackage{enumerate}
%\captionsetup[table]{skip=10pt}

\input{../../library/style}
\addbibresource{../../library/bibliography.bib}

\begin{document}

\chapter{Homework 2}
The connection between modulation and coding is that they are both methods to increase the accuracy of transmission over a noisy channel.
Coding could also refer to line coding, a method of representing information into different line levels. 
%sauce:http://cseweb.ucsd.edu/users/varghese/TEACH/cs123/dl2.pdf
Errors can be divided into two general classes: intersymbol interference (e.g. single errors) and noise. Important causes of errors are: thermal noise and incorrect synchronization at the physical layer, leading to burst errors (e.g. multiple errors after each other).

Q7:
Important considerations when choosing a coding method are: spectrum of coding method must match available baseband channel, availability of self-synchronization, the BEP is acceptably low, available bandwidth is sufficient.
Question 8:
Interleaving means you send the information in a `shuffled' order over the line. After transmission the signal is deinterleaved, or placed back in order. The main purpose of this method is burst error correction, since errors occuring adjacent to each other in transmission are unlikely to be adjacent after deinterleaving.
\end{document}