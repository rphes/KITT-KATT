%!TEX program = xelatex

\documentclass[11pt,titlepage]{report}
%!TEX root = main.tex

\usepackage[T1]{fontenc}
\usepackage{lmodern}
\usepackage[svgnames]{xcolor}
\usepackage{fontspec} % XeLaTeX required!
\usepackage{graphicx}
\usepackage{tikz}
\usepackage{pifont}
\usepackage[some]{background}
\usepackage{xltxtra} 
\usepackage{setspace}
\usepackage[absolute]{textpos}
\usepackage[latin1]{inputenc}
\usepackage[english]{babel}
\usepackage{graphicx}
\usepackage{wrapfig}
\usepackage{fullpage}
\usepackage[margin=1in]{geometry}
\usepackage{float}
\usepackage{url}
\usepackage{multicol}
\usepackage{hyperref}
\usepackage{titlepic}
\usepackage{standalone}
\usepackage{siunitx}
\usepackage{booktabs}
\usepackage{amsmath}
\usepackage{unicode-math}
\usepackage{verbatim}
\usepackage{enumitem}
\usepackage{listings}
\usepackage{multirow}
\usepackage{pgfplots}
\pgfplotsset{compat=1.8}
\usepackage{caption} 
\usepackage[parfill]{parskip}
\usepackage{import}
\usepackage[backend=bibtexu,texencoding=utf8,bibencoding=utf8,style=ieee,sortlocale=en_GB,language=auto]{biblatex}
\usepackage[strict,autostyle]{csquotes}
\usepackage{pdfpages}
%\usepackage{enumerate}
%\captionsetup[table]{skip=10pt}

\input{../../library/style}
\addbibresource{../../library/bibliography.bib}

\begin{document}

\chapter{Introduction}
For the EPO-4 project a Bluetooth controlled vehicle (a modified RC car) must be charged wirelessly, drive through an obstacle course without collision in an autonomous fashion. The basic setup consists of a PC connected with a Bluetooth interface to the car. The car is powered by several super capacitors, charged by magnetic induction through two linked coils, one mounted on the car and one in a base station on the ground. The car can drive using PWM motors controlled via the Bluetooth connection. In order to sense it's surroundings, the car has also been equipped with a pair of distance sensors mounted frontwards on the car, using acoustic waves to measure the proximity to objects.\\
Because the system described above is very complex, steps need to be undertaken to split the design of the overall system in manageable steps. The first identified step was powering the car. Given the super capacitors, the task was to design two linked inductors spaced around 6 cm apart to charge them. The complete design has been discussed in the first deliverable report, and our further steps are described in the first chapter.\\
The second chapter, analogous to the second deliverable, presents our findings on the acoustic sensors used for object detection and avoidance. It gives an overview of the steps taken to reduce measurement errors and serves as a gateway into the third chapter on control of the car, where the output of the sensors are crucial to determining what control strategy to deploy in our car. Further, our findings on the limits of Bluetooth are discussed. Having discussed the more mathematical background of the control system in the third deliverable, the goal of the third chapter is to give insight into the next level of issues we faced, along with some solutions.\\
Before concluding on all findings in the last chapter, the fourth chapter provides an overview of the total system, integrated as one. Because each subsystem was designed in the grander scheme of things, many possible pain-points were eliminated before they could arise. This however, is no guarantee itself for a working car, and our findings are listed here.

\end{document}