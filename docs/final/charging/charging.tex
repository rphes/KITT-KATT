%!TEX program = xelatex

\documentclass[11pt,titlepage]{report}
%!TEX root = main.tex

\usepackage[T1]{fontenc}
\usepackage{lmodern}
\usepackage[svgnames]{xcolor}
\usepackage{fontspec} % XeLaTeX required!
\usepackage{graphicx}
\usepackage{tikz}
\usepackage{pifont}
\usepackage[some]{background}
\usepackage{xltxtra} 
\usepackage{setspace}
\usepackage[absolute]{textpos}
\usepackage[latin1]{inputenc}
\usepackage[english]{babel}
\usepackage{graphicx}
\usepackage{wrapfig}
\usepackage{fullpage}
\usepackage[margin=1in]{geometry}
\usepackage{float}
\usepackage{url}
\usepackage{multicol}
\usepackage{hyperref}
\usepackage{titlepic}
\usepackage{standalone}
\usepackage{siunitx}
\usepackage{booktabs}
\usepackage{amsmath}
\usepackage{unicode-math}
\usepackage{verbatim}
\usepackage{enumitem}
\usepackage{listings}
\usepackage{multirow}
\usepackage{pgfplots}
\pgfplotsset{compat=1.8}
\usepackage{caption} 
\usepackage[parfill]{parskip}
\usepackage{import}
\usepackage[backend=bibtexu,texencoding=utf8,bibencoding=utf8,style=ieee,sortlocale=en_GB,language=auto]{biblatex}
\usepackage[strict,autostyle]{csquotes}
\usepackage{pdfpages}
%\usepackage{enumerate}
%\captionsetup[table]{skip=10pt}

\input{../../library/style}
\addbibresource{../../library/bibliography.bib}

\begin{document}

\chapter{Contactless Charging}
In this chapter it will be briefly described how a contactless charging system was designed and built. The main idea to achieve this was by transmitting energy through space by using magnetic fields. In this system one coil is a transmitter of magnetic energy and the second coil a receiver. For this a fluctuating magnetic field had to be created. Nevertheless, both the energy source and the capacitor bank of the car use direct current. This problems thus translates into designing a DC/DC converter. Furthermore, this problem can be dissected in designing a DC/AC-inverter, a transformer and a AC/DC-rectifier. This charging system can be seen in Figure~\ref{fig:contactless-charging}.

\begin{figure}[H]
	\begin{center}
		\includegraphics[width=0.8\linewidth]{resource/contactless_charging.png}
	\end{center}
	\caption{The wireless charging system with supercapacitors}
	\label{fig:contactless-charging}
\end{figure}

The DC/AC-inverter was made by making use of power electronics. This DC/AC-inverter converts a DC signal into a square wave signal. This signal had to be in solid resonance, for this we used a frequency of \SI{100}{kHz}. For optimal charging speed we used a duty cycle of \SI{40}{\percent}. After this signal is transmitted to the secondary coil it is then converted to a DC signal by using diodes. The parameters of the coils are displayed in Table~\ref{tab:charging-coil-params-calc}.

\begin{table}[H]
	\centering
	\caption{Calculated coil parameters}
	\label{tab:charging-coil-params-calc}
	\begin{tabular}{c c c c c}
		\hline\hline
		Coil & Windings & Inductance & Outer diameter & Total wire length \\
		\hline
		Primary & \num{30} & \SI{101.6}{\micro H} & \SI{188}{mm} & \SI{11.2}{m} \\
		Secondary & \num{15} & \SI{22}{\micro H} & \SI{119}{mm} & \SI{4}{m} \\
		\hline
		\end{tabular}
\end{table} 

This system gave us a maximum charging speed of 3:45 minutes and met all the requirements.

\end{document}