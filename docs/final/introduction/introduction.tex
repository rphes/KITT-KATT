%!TEX program = xelatex

\documentclass[11pt,titlepage]{report}
%!TEX root = main.tex

\usepackage[T1]{fontenc}
\usepackage{lmodern}
\usepackage[svgnames]{xcolor}
\usepackage{fontspec} % XeLaTeX required!
\usepackage{graphicx}
\usepackage{tikz}
\usepackage{pifont}
\usepackage[some]{background}
\usepackage{xltxtra} 
\usepackage{setspace}
\usepackage[absolute]{textpos}
\usepackage[latin1]{inputenc}
\usepackage[english]{babel}
\usepackage{graphicx}
\usepackage{wrapfig}
\usepackage{fullpage}
\usepackage[margin=1in]{geometry}
\usepackage{float}
\usepackage{url}
\usepackage{multicol}
\usepackage{hyperref}
\usepackage{titlepic}
\usepackage{standalone}
\usepackage{siunitx}
\usepackage{booktabs}
\usepackage{amsmath}
\usepackage{unicode-math}
\usepackage{verbatim}
\usepackage{enumitem}
\usepackage{listings}
\usepackage{multirow}
\usepackage{pgfplots}
\pgfplotsset{compat=1.8}
\usepackage{caption} 
\usepackage[parfill]{parskip}
\usepackage{import}
\usepackage[backend=bibtexu,texencoding=utf8,bibencoding=utf8,style=ieee,sortlocale=en_GB,language=auto]{biblatex}
\usepackage[strict,autostyle]{csquotes}
\usepackage{pdfpages}
%\usepackage{enumerate}
%\captionsetup[table]{skip=10pt}

\input{../../library/style}
\addbibresource{../../library/bibliography.bib}

\begin{document}

\chapter{Introduction}
<<<<<<< HEAD
In EPO-4 a smart electric car called KITT had to be made able to charge wirelessly and drive through a plane while avoiding obstacles. \\
To charge it wirelessly two coils where used. One connected to the soure and one connected to KITT. The main task was to charge KITT's super capacitor bank as quick as possible. \\
To drive KITT via a PC firstly Bluetooth communication has to be established. To localize KITT sound signals were sent by a beacon on KITT. This signals where detected on five microphones and analyzed by the PC. Eventually this information is used to determine the speed and the steering of the car. The car can drive using PWM motors controlled via the Bluetooth connection. Furthermore, KITT was equiped with two acoustic sensors. This sensors where used to determine where obstacles were placed. \\ 



Driving KITT through plane while avoiding obstacles can be seen as one big complex system. The complixity of this system can be reduced by dividing it in sub-components. How the entire system is divided in multiple sub-components is shown in Chapter~\ref{ch:system-overview}.  \\ \\
 In Chapter~\ref{ch:charging} it will be explained how contacless charging was achieved and how it is optimized. Eventually the results of the built charging system are shown. \\ \\
Chapter~\ref{ch:localization} gives information on the signals that were sent and how this signals were processed. Along with mathematical background on channel estimation it is illustrated how localization is achieved.  \\  \\
Thorough information on how our car was controlled by our system can be found in  In Chapter~\ref{ch:control}. Firstly it is explained how our car is controlled on a one dimensional line. From here it explains how the latter solution is used to control our car on a plain, this is what we call \textit{reduction of dimensionality}. \\  \\
Chapter~\ref{ch:gui-communication} will elaborate two sub-components GUI and communication. After integrating the whole system it was tested via simulations. The Chapter~\ref{ch:simulation} is dedication to this. \\  \\
And last but not least conclusions are given in Chapter~\ref{ch:conclusions}.


  


=======
The subject of the EPO-4 project already becomes clear from its name \textit{Electric Transport 2020}. This project focusses on the electric charging and autonomous driving aspect of future electric cars.

It's of great relevance to investigate and optimize the charging systems of electric cars for our future, as more electric cars might subtitute cars using natural resources like oil. For the EPO-4 project a contactless charging system was designed and built. In this document, it will be broadly described how this was achieved. For more thorough information on the charging system consult our earlier deliverable reports. 

Autonomous driving will completely change the way we transport. There are some advantages that are worthwhile mentioning. For example, autonomous driving systems could mean fewer tranfic collisions. In response to the latter advantange, speed limits could be raised and roadway capacity could be increased. As for this project we limited ourselves to localizing the car and letting it drive autonomously while avoiding obstacles. How this was achieved will, be elaborated in this document. Likewise, one can find more information on this topic in the deliverable reports. 
>>>>>>> origin/master


\end{document}