%!TEX program = xelatex

\documentclass[11pt,titlepage]{report}
%!TEX root = main.tex

\usepackage[T1]{fontenc}
\usepackage{lmodern}
\usepackage[svgnames]{xcolor}
\usepackage{fontspec} % XeLaTeX required!
\usepackage{graphicx}
\usepackage{tikz}
\usepackage{pifont}
\usepackage[some]{background}
\usepackage{xltxtra} 
\usepackage{setspace}
\usepackage[absolute]{textpos}
\usepackage[latin1]{inputenc}
\usepackage[english]{babel}
\usepackage{graphicx}
\usepackage{wrapfig}
\usepackage{fullpage}
\usepackage[margin=1in]{geometry}
\usepackage{float}
\usepackage{url}
\usepackage{multicol}
\usepackage{hyperref}
\usepackage{titlepic}
\usepackage{standalone}
\usepackage{siunitx}
\usepackage{booktabs}
\usepackage{amsmath}
\usepackage{unicode-math}
\usepackage{verbatim}
\usepackage{enumitem}
\usepackage{listings}
\usepackage{multirow}
\usepackage{pgfplots}
\pgfplotsset{compat=1.8}
\usepackage{caption} 
\usepackage[parfill]{parskip}
\usepackage{import}
\usepackage[backend=bibtexu,texencoding=utf8,bibencoding=utf8,style=ieee,sortlocale=en_GB,language=auto]{biblatex}
\usepackage[strict,autostyle]{csquotes}
\usepackage{pdfpages}
%\usepackage{enumerate}
%\captionsetup[table]{skip=10pt}

\input{../../library/style}
\addbibresource{../../library/bibliography.bib}

\begin{document}

\chapter{Introduction}
The subject of the EPO-4 project already becomes clear from its name \textit{Electric Transport 2020}. This project focusses on the electric charging and autonomous driving aspect of future electric cars. It's of great relevance to investigate and optimize the charging systems of electric cars for our future, as more electric cars might subtitute cars using natural resources like oil. Autonomous driving will completely change the way we transport. There are some advantages that are worthwhile mentioning. For example, autonomous driving systems could mean fewer traffic collisions. In response to the latter advantange, speed limits could be raised and roadway capacity could be increased. One can find more information on this topic in the deliverable reports. 

The goal of the EPO-4 project is to design a system which is able to wirelessly charge and then autonomously drive a car, named KITT, to a certain location. In Chapter~\ref{ch:charging} explains the wireless charging process. To consequently drive KITT to a certain location, a Bluetooth connection has to be esthablished. A system overview will be given in Chapter~\ref{ch:system-overview}. Chapter~\ref{ch:localization} describes the way KITT will localize itself. Chapter~\ref{ch:control} describes the controller which will be responsible for driving to the given location. The whole system needs to be interfaced with, this will be treated in Chapter~\ref{ch:gui-communication}. System verification is done is Chapter~\ref{ch:simulation}. Finally, the report will be concluded in Chapter~\ref{ch:conclusions}.

\end{document}