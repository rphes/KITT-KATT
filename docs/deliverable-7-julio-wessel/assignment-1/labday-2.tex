%!TEX program = xelatex

\documentclass[11pt,titlepage]{report}
%!TEX root = main.tex

\usepackage[T1]{fontenc}
\usepackage{lmodern}
\usepackage[svgnames]{xcolor}
\usepackage{fontspec} % XeLaTeX required!
\usepackage{graphicx}
\usepackage{tikz}
\usepackage{pifont}
\usepackage[some]{background}
\usepackage{xltxtra} 
\usepackage{setspace}
\usepackage[absolute]{textpos}
\usepackage[latin1]{inputenc}
\usepackage[english]{babel}
\usepackage{graphicx}
\usepackage{wrapfig}
\usepackage{fullpage}
\usepackage[margin=1in]{geometry}
\usepackage{float}
\usepackage{url}
\usepackage{multicol}
\usepackage{hyperref}
\usepackage{titlepic}
\usepackage{standalone}
\usepackage{siunitx}
\usepackage{booktabs}
\usepackage{amsmath}
\usepackage{unicode-math}
\usepackage{verbatim}
\usepackage{enumitem}
\usepackage{listings}
\usepackage{multirow}
\usepackage{pgfplots}
\pgfplotsset{compat=1.8}
\usepackage{caption} 
\usepackage[parfill]{parskip}
\usepackage{import}
\usepackage[backend=bibtexu,texencoding=utf8,bibencoding=utf8,style=ieee,sortlocale=en_GB,language=auto]{biblatex}
\usepackage[strict,autostyle]{csquotes}
\usepackage{pdfpages}
%\usepackage{enumerate}
%\captionsetup[table]{skip=10pt}

\input{../../library/style}
\addbibresource{../../library/bibliography.bib}

% Fancy comb dirac, can be ignored for now. For further use.
%
% \usepackage[OT2,T1]{fontenc}
% \newcommand{\sha}{\text{\fontencoding{OT2}\selectfont\char88}}
% \newcommand{\F}[1]{\operatorname{\mathcal{F}}\left\{#1\right\}}
% \begin{equation}
% 	\F{\operatorname{rect}(F_s t) \ast 2F_s \sha(2F_s t)}
% \end{equation}

\begin{document}

\section{Labday 2}
\subsection{Report 8}
The suitabily of four sequences had to be investigated. The four sequences given where:

\begin{equation}
x_1 = [1,-\frac{1}{2},0,0,....]^T \\ 
222
\end{equation}

\begin{equation}
x_2 = [1,-2,0,0,....]^T
\end{equation}
	
\begin{equation}
x_3[n] = \left\{ 
  \begin{array}{l l}
   cos(0.2n) & \quad \textrm{if $0 \leq n \leq N-1 $}\\
    0 & \quad \textrm{if otherwise}
  \end{array} \right.
\end{equation}
 
\begin{equation}
x_4 = sign(randn(N,1))
\end{equation}

For each of this sequences the autocorrelation was calculated and is displayed in Figure~\ref{fig:rep8-autocor}. From this plots we can see that minimum-phase and maximum-phase behave the most like impulses. Impulse behaviour within the autocorrelation means that the signal can better be detected. Thus we can conclude that minimum-phase and maximum-phase are most suited for our purpose. 

\begin{figure}[H]
	\centering
	\begin{subfigure}{0.49\textwidth}
		\includegraphics[width=\textwidth]{../../deliverable-7-resources/figures/ass-1/report-8-9-10/report-8/ass-1-report-8-minimum-phase-minimum-phase.pdf}
	\end{subfigure}
	\begin{subfigure}{0.49\textwidth}
		\includegraphics[width=\textwidth]{../../deliverable-7-resources/figures/ass-1/report-8-9-10/report-8/ass-1-report-8-maximum-phase-maximum-phase.pdf}
	\end{subfigure}
	\begin{subfigure}{0.49\textwidth}
		\includegraphics[width=\textwidth]{../../deliverable-7-resources/figures/ass-1/report-8-9-10/report-8/ass-1-report-8-sinusoidal-sinusoidal.pdf}
	\end{subfigure}
	\begin{subfigure}{0.49\textwidth}
		\includegraphics[width=\textwidth]{../../deliverable-7-resources/figures/ass-1/report-8-9-10/report-8/ass-1-report-8-BPSK-BPSK.pdf}
	\end{subfigure}
	\caption{Autocorrelation sequences for each of the tested signals}
	\label{fig:rep8-autocor}
\end{figure}
 

\subsection{Report 9}


\begin{figure}[H]
	\centering
	\begin{subfigure}{0.49\textwidth}
		\includegraphics[width=\textwidth]{../../deliverable-7-resources/figures/ass-1/report-8-9-10/report-9-noise-0.5/ass-1-report-9-minimum-phase.pdf}
	\end{subfigure}
	\begin{subfigure}{0.49\textwidth}
		\includegraphics[width=\textwidth]{../../deliverable-7-resources/figures/ass-1/report-8-9-10/report-9-noise-0.5/ass-1-report-9-maximum-phase.pdf}
	\end{subfigure}
	\begin{subfigure}{0.49\textwidth}
		\includegraphics[width=\textwidth]{../../deliverable-7-resources/figures/ass-1/report-8-9-10/report-9-noise-0.5/ass-1-report-9-sinusoidal.pdf}
	\end{subfigure}
	\begin{subfigure}{0.49\textwidth}
		\includegraphics[width=\textwidth]{../../deliverable-7-resources/figures/ass-1/report-8-9-10/report-9-noise-0.5/ass-1-report-9-BPSK.pdf}
	\end{subfigure}
	\caption{The channel estimation error for increasing $\hat{L}$,  with $\sigma = 0.5$ noise}
	\label{fig:rep9-0.5}
\end{figure}

The latter Figure~\ref{fig:rep9-0.5} shows the channel estimation with $\sigma = 0.5$. $\hat{L} \ge L = 4$

\begin{figure}[H]
	\centering
	\begin{subfigure}{0.49\textwidth}
		\includegraphics[width=\textwidth]{../../deliverable-7-resources/figures/ass-1/report-8-9-10/report-9-noise-0.1/ass-1-report-9-minimum-phase.pdf}
	\end{subfigure}
	\begin{subfigure}{0.49\textwidth}
		\includegraphics[width=\textwidth]{../../deliverable-7-resources/figures/ass-1/report-8-9-10/report-9-noise-0.1/ass-1-report-9-maximum-phase.pdf}
	\end{subfigure}
	\begin{subfigure}{0.49\textwidth}
		\includegraphics[width=\textwidth]{../../deliverable-7-resources/figures/ass-1/report-8-9-10/report-9-noise-0.1/ass-1-report-9-sinusoidal.pdf}
	\end{subfigure}
	\begin{subfigure}{0.49\textwidth}
		\includegraphics[width=\textwidth]{../../deliverable-7-resources/figures/ass-1/report-8-9-10/report-9-noise-0.1/ass-1-report-9-BPSK.pdf}
	\end{subfigure}
	\caption{The channel estimation error for increasing $\hat{L}$, with $\sigma = 0.1$ noise}
	\label{fig:rep9-0.1}
\end{figure}





\subsection{Report 10}
Training sequence design
\end{document}