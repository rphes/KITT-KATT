%!TEX program = xelatex

\documentclass[11pt,titlepage]{report}
%!TEX root = main.tex

\usepackage[T1]{fontenc}
\usepackage{lmodern}
\usepackage[svgnames]{xcolor}
\usepackage{fontspec} % XeLaTeX required!
\usepackage{graphicx}
\usepackage{tikz}
\usepackage{pifont}
\usepackage[some]{background}
\usepackage{xltxtra} 
\usepackage{setspace}
\usepackage[absolute]{textpos}
\usepackage[latin1]{inputenc}
\usepackage[english]{babel}
\usepackage{graphicx}
\usepackage{wrapfig}
\usepackage{fullpage}
\usepackage[margin=1in]{geometry}
\usepackage{float}
\usepackage{url}
\usepackage{multicol}
\usepackage{hyperref}
\usepackage{titlepic}
\usepackage{standalone}
\usepackage{siunitx}
\usepackage{booktabs}
\usepackage{amsmath}
\usepackage{unicode-math}
\usepackage{verbatim}
\usepackage{enumitem}
\usepackage{listings}
\usepackage{multirow}
\usepackage{pgfplots}
\pgfplotsset{compat=1.8}
\usepackage{caption} 
\usepackage[parfill]{parskip}
\usepackage{import}
\usepackage[backend=bibtexu,texencoding=utf8,bibencoding=utf8,style=ieee,sortlocale=en_GB,language=auto]{biblatex}
\usepackage[strict,autostyle]{csquotes}
\usepackage{pdfpages}
%\usepackage{enumerate}
%\captionsetup[table]{skip=10pt}

\input{../../library/style}
\addbibresource{../../library/bibliography.bib}

\begin{document}

\chapter{Assignment 2}
\section{Observer design}


\section{Controlling the output}
Consider the state-space model

\begin{align}
	\dot{\vec{x}} &= \mat{A}\vec{x} + \mat{B}\vec{u}, \\
	\vec{y} &= \mat{C} \vec{x}. \label{eq:ass-2-model-output}
\end{align}

Let

\begin{equation}
	\vec{u} = -\mat{K}\vec{x} + \vec{r}
\end{equation}

be a feedback law which renders the considered state-space system asymptotically stable. Substituting $\vec{u}$ yields

\begin{equation} \label{eq:ass-2-derivative}
	\dot{\vec{x}} = (\mat{A} - \mat{B} \mat{K}) \vec{x} + \mat{B} \vec{r}.
\end{equation}

Using the fact that our system is asymptotically stable, we can argue that $\dot{\vec{x}} \to 0$ when $t \to \infty$. Combining Equation~\ref{eq:ass-2-model-output} and \ref{eq:ass-2-derivative} yields

\begin{equation}
	\vec{y} \to \mat{C} (\mat{B} \mat{K} - \mat{A})^{-1} \mat{B} \vec{r} \text{ when } t \to \infty.
\end{equation}

Therefore, the output $\vec{y}$ converges to the scaled applied input $\vec{r}$. We can utilize this fact to control the output of the model.

\section{The controller}
Putting together the compensator, observer and output reference leaves us at the controller given by

\begin{align}
	\dot{\hat{\vec{x}}} &= (\mat{A}-\mat{L}\mat{C}-\mat{B}\mat{K}) \hat{\vec{x}} + \mat{B}(\mat{C}(\mat{B} \mat{K} - \mat{A})^{-1} \mat{B})^{-1} \vec{r} + \mat{L} \vec{y}, \\
	\vec{x} &= (\mat{C}(\mat{B} \mat{K} - \mat{A})^{-1} \mat{B})^{-1} \vec{r} - \mat{K} \hat{\vec{x}}.
\end{align}

Here the drive excitation is denoted by $\vec{x}$ and distance measured by $\vec{y}$. However, we can only regulate the throttle of KITT approximately once per \SI{300}{ms}. We must therefore discretize the controller. Discretization using Heun's method \cite{wikipedia-heuns} yields

\begin{align}
	f(\vec{x},\vec{r},\vec{y}) &= (\mat{A}-\mat{L}\mat{C}-\mat{B}\mat{K}) \hat{\vec{x}} + \mat{B}(\mat{C}(\mat{B} \mat{K} - \mat{A})^{-1} \mat{B})^{-1} \vec{r} + \mat{L} \vec{y}, \\
	\hat{\vec{x}}_n &= \hat{\vec{x}}_{n-1} + \frac{T}{2} \left( f\left(\vec{x}_{n-1},\vec{r}_{n-1},\vec{y}_{n-1} \right) + f\left( \vec{x}_{n-1} + T f(\vec{x}_{n-1},\vec{r}_{n-1},\vec{y}_{n-1}),\vec{r}_{n},\vec{y}_{n}\right) \right), \\
	\vec{x}_n &= (\mat{C}(\mat{B} \mat{K} - \mat{A})^{-1} \mat{B})^{-1} \vec{r}_n - \mat{K} \hat{\vec{x}}_n.
\end{align}

Heun's method averages the slopes at points $n$ and $n+1$ using Euler's method to predict the value at point $n+1$. This gives us a decent and relatively easy to implement approximation.

\section{Oscillatory and critically damped response}
The response consists of two exponential functions, whose exponents are the eigenvalues. Therefore, to achieve an oscillatory response, the eigenvalues must be complex conjugated.

The situation is somewhat more complex regarding a perfect critically damped response. The solution of a state-space model is dependent of $e^{\mat{A} t}$, which can be written as

\begin{equation}
	e^{\mat{A} t} = \sum_{k=0}^{\infty} \frac{\mat{A}^k t^k}{k!}.
\end{equation}

If $\mat{A}$ is diagonalizable, then

\begin{equation}
	e^{\mat{A} t} = P \begin{bmatrix}
		e^{\lambda_1 t} & 0 & & 0 \\
		0 & e^{\lambda_2 t} & & \\
		 & & \ddots & 0 \\
		 0 & & 0 & e^{\lambda_n t}
	\end{bmatrix} P^{-1}.
\end{equation}

This calculation yields us an overdamped response. We can therefore roughly state that, if $\mat{A}$ is diagonalizable, this corresponds with an overdamped response.

$e^{\mat{A} t}$ can also be calculated using the $\mat{A}$'s Jordan form. If $e^{\mat{A} t}$ is diagonalizable, then this Jordan form reduces to a regular diagonalization. However, if $e^{\mat{A} t}$ is not diagonalizable, then calculating $e^{\mat{A} t}$ using $\mat{A}$'s Jordan form turns out to give us a critically damped solution \cite{jordan-solution}. Therefore, if we were able to chose a feedback which would render $\mat{A} - \mat{B} \mat{K}$ not diagonalizable, then we would find a critically damped response.

Let us consider the case of a matrix $\mat{A} \in \mathbb{M}_{22}$. If $\lambda$ was to be an eigenvalue, then solving $\operatorname{null}{(\mat{A} - \lambda \mat{I})}$ would yield the corresponding eigenvectors. In general, $\operatorname{nullity}{(\mat{M})} + \operatorname{rank}{(\mat{M})} = n$. Therefore, in our case, $2 - \operatorname{rank}{(\mat{A} - \lambda \mat{I})} = \operatorname{nullity}{(\mat{A} - \lambda \mat{I})}$. However, if $\operatorname{rank}{(\mat{A} - \lambda \mat{I})} \ge 1$, which is the case if $\mat{A} - \lambda \mat{I} \neq \vec{0}$, then the dimension of the eigenspace corresponding with $\lambda$ can not equal two. Therefore, if $\lambda$ would have algebraic multiplicity two, then $\mat{A}$ would be not diagonalizable.

Combining the facts that \textbf{(1)} having a algebraic multiplicity of two renders $\mat{A}$ not diagonalizable, and \textbf{(2)} $\mat{A}$ not being diagonalizble leads us to a critically damped response, allows us to argue that chosing two eigenvalues $\lambda_1$ and $\lambda_2$ with $|\lambda_1- \lambda_2|$ minimal, allows us to mimic critically damped behaviour. 


\end{document}